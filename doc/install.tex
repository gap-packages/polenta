%%%%%%%%%%%%%%%%%%%%%%%%%%%%%%%%%%%%%%%%%%%%%%%%%%%%%%%%%%%%%%%%%%%%%%%%%
%%
%W  install.tex           POLENTA documentation            Bjoern Assmann
%W                                                     
%W                                                     
%W                                                       
%%
%H  @(#)$Id$
%%
%Y 2003
%%
%%%%%%%%%%%%%%%%%%%%%%%%%%%%%%%%%%%%%%%%%%%%%%%%%%%%%%%%%%%%%%%%%%%%%%%%%
\Chapter{Installing and loading the Polenta package}

\atindex{Installing and loading the Polenta package}{@installing
                        and loading the {\Polenta} package}
%%%%%%%%%%%%%%%%%%%%%%%%%%%%%%%%%%%%%%%%%%%%%%%%%%%%%%%%%%%%%%%%%%%%%%%%%
\Section{Installing the Polenta package}\null

\atindex{Installing the Polenta package}{@installing the {\Polenta} package}
The installation of the {\Polenta} package follows standard {\GAP} rules.
So the standard method is to unpack	 the package into the `pkg'
directory  of your {\GAP} distribution.  This will create a `polenta'
subdirectory. 

For other non-standard options please see  Chapter~"ref:Installing a
GAP Package" in the {\GAP} Reference Manual.

Note that the GAP-Packages Alnuth and Polycyclic are needed for this package.
They can be obtained at
\smallskip
   \centerline{\tt  http://cayley.math.nat.tu-bs.de/software/content.html}

%To create the documentation, go into the `doc' directory and type
%`make_doc'.

%%%%%%%%%%%%%%%%%%%%%%%%%%%%%%%%%%%%%%%%%%%%%%%%%%%%%%%%%%%%%%%%%%%%%%%%%
\Section{Loading the Polenta package}\null

\atindex{Loading the Polenta package}{@loading the {\Polenta} package}
If the {\Polenta} Package is not already loaded 
then you have to request it explicitly. 
This  can be 
done by `LoadPackage("polenta")'.
The `LoadPackage' command is described in Section~"ref:LoadPackage"
in the {\GAP} Reference Manual.

%%%%%%%%%%%%%%%%%%%%%%%%%%%%%%%%%%%%%%%%%%%%%%%%%%%%%%%%%%%%%%%%%%%%%%%%%
%%
%E


